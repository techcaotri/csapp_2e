\documentclass[11pt]{article}

\usepackage{times}
\usepackage{alltt}


%% Page layout
\oddsidemargin 0pt
\evensidemargin 0pt
\textheight 600pt
\textwidth 469pt
\setlength{\parindent}{0em}
\setlength{\parskip}{1ex}

%% ccode environment- for displaying formatted C code (c2tex) 
\newenvironment{ccode}%
{\small}%
{}

%% tty - for displaying TTY input and output
\newenvironment{tty}%
{\small\begin{alltt}}%
{\end{alltt}}

\begin{document}

\title{CS 213, Fall 2002\\
Lab Assignment L3: The Buffer Bomb\\
Assigned: Sep.~26, Due: Thurs., Oct.~3, 11:59PM
}

\author{}
\date{}

\maketitle

Randy Bryant ({\tt Randy.Bryant@cs.cmu.edu}) is the lead person for
this assignment.

\section*{Introduction}

This assignment helps you develop a detailed understanding of the
calling stack organization on an IA32 processor.  It involves applying
a series of {\em buffer overflow attacks} on an executable file {\tt
bufbomb} in the lab directory.

{\bf Note:} In this lab, you will gain firsthand experience with one
of the methods commonly used to exploit security weaknesses in
operating systems and network servers.  Our purpose is to help you
learn about the runtime operation of programs and to understand the
nature of this form of security weakness so that you can avoid it
when you write system code.  We do not condone the use of these or any other
form of attack to gain unauthorized access to any system resources.
There are criminal statutes governing such activities.

\section*{Logistics}

You may work in a group of up to two people in solving the problems
for this assignment.  The only ``hand-in'' will be an automated
logging of your successful attacks.  Any clarifications and revisions
to the assignment will be posted on the course Web page.

\section*{Hand Out Instructions}

\begin{quote}
\bf SITE-SPECIFIC: Insert a paragraph here that explains how the instructor
will hand out the \texttt{buflab-handout.tar} file to the students. 
\end{quote}

Start by copying \texttt{buflab-handout.tar} to a (protected) directory
in which you plan to do your work.  Then give the command
``\verb@tar xvf buflab-handout.tar@''.  This will cause a number of files
to be unpacked in the directory:
\begin{description}
\item[{\sc makecookie}:] Generates a ``cookie'' based on your team name.
\item[{\sc bufbomb}:] The code you will attack.
\item[{\sc sendstring}:] A utility to help convert between string formats.
\end{description}
All of these programs are compiled to run on Linux machines.

In the following instructions, we will assume that you have copied the three
programs to a protected local directory, and that you are executing them in
that local directory.


\section*{Team Name and Cookie}

\begin{quote}
NOTE TO INSTRUCTORS: In this section, we are assuming that students
have a unique login ID. At our school this is called an ``Andrew ID.''
You will want to replace this with something appropriate for your
school, such as ``login name.''
\end{quote}

You should create a team name for the one or two people in your group
of the following form:
\begin{itemize}
\item ``${\it ID}$'' where ${\it ID}$ is your Andrew ID, if you are
working alone, or
\item ``${\it ID}_1${\tt +}${\it ID}_2$'' where ${\it ID}_1$ is the
Andrew ID of the first team member and ${\it ID}_2$ is the Andrew ID
of the second team member.  
\end{itemize}
You should choose a consistent ordering of the IDs in the second form
of team name.  Teams ``{\tt ac00+bovik}'' and ``{\tt bovik+ac00}'' are
considered distinct.  {\bf You must follow this scheme for generating
your team name.  Our grading program will only give credit to those
people whose Andrew IDs can be extracted from the team names.}

A {\em cookie} is a string of eight hexadecimal digits that is (with high
probability) unique to your team.  You can generate your cookie with
the {\tt makecookie} program giving your team name as the argument.
For example:
\begin{tty}
unix>{\em ./makecookie ac00+bovik}
0x78327b66
\end{tty}
In four of your five buffer attacks, your objective will be to make
your cookie show up in places where it ordinarily would not.

\section*{The {\sc Bufbomb} Program}

The {\sc bufbomb} program reads a string from standard input with a
function {\tt getbuf} having the following C code:

\begin{ccode}
\begin{alltt}
{\scriptsize   1} int getbuf()
{\scriptsize   2} \verb:{:
{\scriptsize   3}     char buf[12];
{\scriptsize   4}     Gets(buf);
{\scriptsize   5}     return 1;
{\scriptsize   6} \verb:}:
\end{alltt}

\end{ccode}

The function {\tt Gets} is similar to the standard library function
{\tt gets}---it reads a string from standard input (terminated by
`\verb@\n@' or end-of-file) and stores it (along with a null
terminator) at the specified destination.  In this code, the
destination is an array {\tt buf} having sufficient space for 12
characters.

Neither {\tt Gets} nor {\tt gets} has any way to determine whether
there is enough space at the destination to store the entire string.
Instead, they simply copy the entire string, possibly overrunning the
bounds of the storage allocated at the destination.

If the string typed by the user to {\tt getbuf} is no more than 11
characters long, it is clear that {\tt getbuf} will return 1, as shown
by the following execution example:

\begin{tty}
unix>{\em ./bufbomb}
Type string:{\em howdy doody}
Dud: getbuf returned 0x1
\end{tty}

Typically an error occurs if we type a longer string:

\begin{tty}
unix>{\em ./bufbomb}
Type string:{\em This string is too long}
Ouch!: You caused a segmentation fault!
\end{tty}

As the error message indicates, overrunning the buffer typically
causes the program state to be corrupted, leading to a memory access
error.  Your task is to be more clever with the strings you feed {\sc
bufbomb} so that it does more interesting things.  These are called
{\em exploit} strings.

{\sc Bufbomb} takes several different command line arguments:
\begin{description}
\item[{\tt -t} {\it TEAM}:] Operate the bomb for the indicated team.
You should always provide this argument for several reasons:
\begin{itemize}
\item It is required to log your successful attacks.

\item {\sc Bufbomb} determines the cookie you will be using based on
your team name, just as does the program {\sc makecookie}.

\item
We have built features into {\sc bufbomb} so that some of the key
stack addresses you will need to use depend on your team's cookie.
\end{itemize}
\item[{\tt -h}:] Print list of possible command line arguments
\item[{\tt -n}:] Operate in ``Nitro'' mode, as is used in Level 4 below.
\end{description}

Your exploit strings will typically contain byte values that do not
correspond to the ASCII values for printing characters.  The program
{\sc sendstring} can help you generate these {\em raw} strings.  It
takes as input a {\em hex-formatted} string.  In this format, each
byte value is represented by two hex digits.  For example, the string
``{\tt 012345}'' could be entered in hex format as ``{\tt 30 31 32 33
34 35}.'' (Recall that the ASCII code for decimal digit $x$ is {\tt 0x3}$x$.)
Non-hex digit characters are ignored, including the blanks in the
example shown.

If you generate a hex-formatted exploit string in the file {\tt
exploit.txt}, you can apply the raw string to {\sc bufbomb} in several
different ways:
\begin{enumerate}
\item You can set up a series of pipes to pass the string through {\sc sendstring}.
\begin{tty}
unix>{\em cat exploit.txt | ./sendstring | ./bufbomb -t bovik}
\end{tty}
\item
You can store the raw string in a file and use I/O redirection to supply it to {\sc bufbomb}:
\begin{tty}
unix>{\em ./sendstring < exploit.txt > exploit-raw.txt}
unix>{\em ./bufbomb -t bovik < exploit-raw.txt}
\end{tty}
This approach can also be used when running {\sc bufbomb} from within
{\sc gdb}:
\begin{tty}
unix>{\em gdb bufbomb}
(gdb){\em run -t bovik < exploit-raw.txt}
\end{tty}
\end{enumerate}

One important point: your exploit string must not contain byte value
{\tt 0x0A} at any intermediate position,
since this is the ASCII code for newline (`\verb@\n@').
When {\tt Gets} encounters this byte, it will assume you intended to
terminate the string.  {\sc Sendstring} will warn you if it encounters
this byte value.

When you correctly solve one of the levels, {\sc bufbomb}
will automatically send an email notification to our grading server.
The server will test your exploit string to make sure it really works,
and it will update the lab web page indicating that your team (listed
by cookie) has completed this level.

Unlike the bomb lab, there is no penalty for making mistakes in this
lab.  Feel free to fire away at {\sc bufbomb} with any string you
like.

\section*{Level 0: Candle (10 pts)}

The function {\tt getbuf} is called within {\sc bufbomb} by a function
{\tt test} having the following C code:

\begin{ccode}
\input boom-c
\end{ccode}

When {\tt getbuf} executes its return statement (line 5 of {\tt
getbuf}), the program ordinarily resumes execution within function
{\tt test} (at line 8 of this function).
Within the file {\tt bufbomb}, there is a function {\tt smoke} having
the following C code:

\begin{ccode}
\input smoke-c
\end{ccode}

Your task is to get {\sc bufbomb} to execute the code for {\tt smoke}
when {\tt getbuf} executes its return statement, rather than returning
to {\tt test}.  You can do this by supplying an exploit string that
overwrites the stored return pointer in the stack frame for {\tt
getbuf} with the address of the first instruction in {\tt smoke}.
Note that your exploit string may also corrupt other parts of the
stack state, but this will not cause a problem, since {\tt smoke}
causes the program to exit directly.

{\bf Some Advice}:
\begin{itemize}
\item All the information you need to devise your exploit
string for this level can be determined by examining a diassembled
version of {\sc bufbomb}.
\item
Be careful about byte ordering.  
\item
You might
want to use {\sc gdb} to step the program through the last few
instructions of {\tt getbuf} to make sure it is doing the right thing.
\item
The placement of {\tt buf} within the stack frame for {\tt getbuf}
depends on which version of {\sc gcc} was used to compile {\tt
bufbomb}.  You will need to pad the beginning of your exploit string
with the proper number of bytes to overwrite the return pointer.  The
values of these bytes can be arbitrary.
\end{itemize}

\section*{Level 1: Sparkler (20 pts)}

Within the file {\tt bufbomb} there is also a function {\tt fizz}
having the following C code:

\begin{ccode}
\input fizz-c
\end{ccode}

Similar to Level 0, your task is to get {\sc bufbomb} to execute the
code for {\tt fizz} rather than returning to {\tt test}.  In this
case, however, you must make it appear to {\tt fizz} as if you have
passed your cookie as its argument.  You can do this by encoding your
cookie in the appropropriate place within your exploit string.

{\bf Some Advice}: 
\begin{itemize}
\item Note that the program won't really call {\tt
fizz}---it will simply execute its code.  This has important
implications for where on the stack you want to place your
cookie.
\end{itemize}

\section*{Level 2: Firecracker (30 pts)}

A much more sophisticated form of buffer attack involves supplying a string
that encodes actual machine instructions.  The exploit string then
overwrites the return pointer with the starting address of these instructions.
When the calling function (in this case {\tt getbuf}) executes its {\tt
ret} instruction, the program will start executing the instructions on
the stack rather than returning.  With this form of attack, you can get
the program to do almost anything.  The code you place on the stack is
called the {\em exploit} code.  This style of attack is tricky,
though, because you must get machine code onto the stack and set the
return pointer to the start of this code.

Within the file {\tt bufbomb} there is a function {\tt bang}
having the following C code:

\begin{ccode}
\input bang-c
\end{ccode}

Similar to Levels 0 and 1, your task is to get {\sc bufbomb} to
 execute the code for {\tt bang} rather than returning to {\tt test}.
 Before this, however, you must set global variable
 \verb@global_value@ to your team's cookie.  Your exploit code should
 set \verb@global_value@, push the address of \texttt{bang} on the
 stack, and then execute a \texttt{ret} instruction to cause a jump to
 the code for \texttt{bang}.

{\bf Some Advice}:
\begin{itemize}
\item 
You can use {\sc gdb} to get the information you need to construct
your exploit string.  Set a breakpoint within {\tt getbuf} and run to
this breakpoint.  Determine parameters such as the address of \verb@global_value@ and the location of the buffer.
\item
Determining the byte encoding of instruction sequences by hand is
tedious and prone to errors.  You can let tools do all of the work by
writing an assembly code file containing the instructions and data you
want to put on the stack.  Assemble this file with {\sc gcc} and
disassemble it with {\sc objdump}.  You should be able to get the
exact byte sequence that you will type at the prompt.  
(A brief example of how to do this is included at the end of this writeup.)
\item
Keep in mind that your exploit string depends on your machine, your
compiler, and even your team's cookie.  Do all of your work on a Fish
machine, and make sure you include the proper team name on the command line
to {\sc bufbomb}.
\item
Our solution requires 16 bytes of exploit code.  Fortunately, there is
sufficient space on the stack, because we can overwrite the stored
value of \verb@%ebp@.  This stack corruption will not cause any
problems, since \texttt{bang} causes the program to exit directly.
\item
Watch your use of address modes when writing assembly code.
Note that \verb@movl $0x4, %eax@ moves the {\em value} \texttt{0x00000004}
into register \verb@%eax@; whereas \verb@movl 0x4, %eax@ moves the value
{\em at} memory location \texttt{0x00000004} into \verb@%eax@. Since that
memory location is usually undefined, the second instruction will cause a
segfault!
\item
Do not attempt to use either a \texttt{jmp} or a \texttt{call}
instruction to jump to the code for \texttt{bang}.  These instructions
uses PC-relative addressing, which is very tricky to set up correctly.
Instead, push an address on the stack and use the \texttt{ret}
instruction.
\end{itemize}


\section*{Level 3: Dynamite (40 pts)}

Our preceding attacks have all caused the program to jump to the code
for some other function, which then causes the program to exit.  As a
result, it was acceptable to use exploit strings that corrupt the
stack, overwriting the saved value of register \verb@%ebp@ and the
return pointer.  

The most sophisticated form of buffer overflow attack causes the
program to execute some exploit code that patches up the stack and
makes the program return to the original calling function
(\texttt{test} in this case).  The calling function is oblivious to
the attack.  This style of attack is tricky, though, since you must:
1) get machine code onto the stack, 2) set the return pointer to the
start of this code, and 3) undo the corruptions made to the stack
state.

Your job for this level is to supply an exploit string that will cause
{\tt getbuf} to return your cookie back to \texttt{test}, rather than
the value 1.  You can see in the code for {\tt test} that this will
cause the program to go ``{\tt Boom!}.''  Your exploit code should set
your cookie as the return value, restore any corrupted state, push the
correct return location on the stack, and execute a {\tt ret}
instruction to really return to {\tt test}.

{\bf Some Advice}:
\begin{itemize}
\item In order to overwrite the return pointer, you must also
overwrite the saved value of \verb@%ebp@.  However, it is important
that this value is correctly restored before you return to {\tt test}.
You can do this by either 1) making sure that your exploit string
contains the correct value of the saved \verb@%ebp@ in the correct
position, so that it never gets corrupted, or 2) restore the correct
value as part of your exploit code.  You'll see that the code for
\texttt{test} has some explicit tests to check for a corrupted stack.
\item 
You can use {\sc gdb} to get the information you need to construct
your exploit string.  Set a breakpoint within {\tt getbuf} and run to
this breakpoint.  Determine parameters such as the saved return
address and the saved value of \verb@%ebp@.
\item
Again, let tools such as \textsc{gcc} and \textsc{objdump} do all of
the work of generating a byte encoding of the instructions.
\item
Keep in mind that your exploit string depends on your machine, your
compiler, and even your team's cookie.  Do all of your work on a Fish
machine, and make sure you include the proper team name on the command line
to {\sc bufbomb}.
\end{itemize}

Once you complete this level, pause to reflect on what you have
accomplished.  You caused a program to execute machine code of your
own design.  You have done so in a sufficiently stealthy way that the program
did not realize that anything was amiss.

\section*{Level 4: Nitroglycerin (10 pts)}

If you have completed the first four levels, you have earned 100
points.  You have mastered the principles of the runtime stack
operation, and you have gained firsthand experience with buffer
overflow attacks.  We consider this a satisfactory mastery of the
material.  You are welcome to stop right now.

The next level is for those who want to push themselves beyond our
baseline expectations for the course, and who want to face a challenge
in designing buffer overflow attacks that arises in real life.  This
part of the assignment only counts 10 points, even though it requires
a fair amount of work to do, so don't do it just for the points.

From one run to another, especially by different users, the exact
stack positions used by a given procedure will vary.  One reason for
this variation is that the values of all environment variables are
placed near the base of the stack when a program starts executing.
Environment variables are stored as strings, requiring different
amounts of storage depending on their values.  Thus, the stack space
allocated for a given user depends on the settings of his or her
environment variables.  Stack positions also differ when running a
program under {\sc gdb}, since {\sc gdb} uses stack space for some of its
own state.

In the code that calls {\tt getbuf}, we have incorporated features
that stabilize the stack, so that the position of {\tt getbuf}'s stack
frame will be consistent between runs.  This made it possible for you
to write an exploit string knowing the exact starting address of {\tt
buf} and the exact saved value of \verb@%ebp@.  If you tried to use
such an exploit on a normal program, you would find that it works some
times, but it causes segmentation faults at other times.  Hence the
name ``dynamite''---an explosive developed by Alfred Nobel that
contains stabilizing elements to make it less prone to unexpected
explosions.

For this level, we have gone the opposite direction, making the stack
positions even less stable than they normally are.  Hence the name
``nitroglycerin''---an explosive that is notoriously unstable.

When you run {\sc bufbomb} with the command line flag ``{\tt -n},'' it
will run in ``Nitro'' mode.  Rather than calling the function {\tt
getbuf}, the program calls a slightly different function {\tt
getbufn}:

\begin{ccode}
\input kaboom-c
\end{ccode}

This function is similar to {\tt getbuf}, except that it has a buffer
of 512 characters.  You will need this additional space to create a
reliable exploit.  The code that calls {\tt getbufn} first allocates a
random amount of storage on the stack (using library function {\tt
alloca}) that ranges between 0 and 127 bytes.  Thus, if you were to
sample the value of \verb@%ebp@ during two successive executions of
{\tt getbufn}, you would find they differ by as much as $\pm 127$.

In addition, when run in Nitro mode, {\sc bufbomb} requires you to
supply your string 5 times, and it will execute {\tt getbufn} 5 times,
each with a different stack offset.  Your exploit string must make it
return your cookie each of these times.

Your task is identical to the task for the Dynamite level. Once again,
your job for this level is to supply an exploit string that will cause
\texttt{getbufn} to return your cookie back to test, rather than the value 1.
You can see in the code for test that this will cause the program to go
``{\tt KABOOM!}.'' Your exploit code should set your cookie as the return
value, restore any corrupted state, push the correct return location on
the stack, and execute a \verb@ret@ instruction to really return to
{\tt testn}.

{\bf Some Advice}:
\begin{itemize}
\item You can use the program {\sc sendstring} to send multiple copies
of your exploit string. 
 If you have a single copy in the file {\tt exploit.txt}, then you can 
use the following command:

\begin{tty}
unix>{\em cat exploit.txt | ./sendstring -n 5 | ./bufbomb -n -t bovik}
\end{tty}

You must use the same string for all 5 executions of {\tt getbufn}.
Otherwise it will fail the testing code used by our grading server.

\item
The trick is to make use of the {\tt nop} instruction.  It is encoded with
a single byte (code {\tt 0x90}).  You can place a long sequence of
these at the beginning of your exploit code so that your code will work
correctly if the initial jump lands anywhere within the sequence.

\item
You will need to restore the saved value of \verb@%ebp@ in a way that
is insensitive to variations in stack positions.

\end{itemize}

\section*{Logistical Notes}

Hand in occurs automatically whenever you correctly solve a level.
The program sends email to our grading server containing your team
name (be sure to set the ``{\tt -t}'' command line flag properly) and
your exploit string to the grading server.  You will be informed of
this by {\sc bufbomb}.  Upon receiving the email, the server will
validate your string and update the lab web page.  You should check
this page a few minutes after your submission to make sure your string
has been validated.  [If you really solved the level, your string {\em
should} be valid.]

Note that each level is graded individually.  You do not need to do
them in the specified order, but you will get credit only for the
levels for which the server receives a valid message.

Have fun!

\section*{Generating Byte Codes}

Using {\sc gcc} as an assembler and {\sc objdump} as a disassembler
makes it convenient to generate the byte codes for instruction sequences.
For example, suppose we write a file \verb@example.s@ containing the
following assembly code:
\begin{tty}
\input{example.s}
\end{tty}
The code can contain a mixture of instructions and data.
Anything to the right of a `\verb@#@' character is a comment.  
We have
added an extra word of all 0s to work around a shortcoming in {\sc
objdump} to be described shortly.

We can now assemble and disassemble this file:
\begin{tty}
unix>{\em gcc -c example.s}
unix>{\em objdump -d example.o > example.d}
\end{tty}
The generated file {\tt example.d} contains the following lines
\begin{ccode}
\begin{alltt}
   0:   68 ef cd ab 89          push   $0x89abcdef{\em\scriptsize }
   5:   83 c0 11                add    $0x11,%eax{\em\scriptsize }
   8:   98                      cwtl                {\em\scriptsize  Objdump tries to interpret}
   9:   ba dc fe 00 00          mov    $0xfedc,%edx {\em\scriptsize  these as instructions}
\end{alltt}

\end{ccode}
Each line shows a single instruction.  The number on the left
indicates the starting address (starting with 0), while the hex digits
after the `\verb@:@' character indicate the byte codes for the
instruction.  Thus, we can see that the instruction {\tt pushl
\$0x89ABCDEF} has hex-formatted byte code {\tt 68 ef cd ab 89}.

Starting at address 8, the disassembler gets confused.  It tries to
interpret the bytes in the file {\tt example.o} as instructions, but
these bytes actually correspond to data.  Note, however, that if we
read off the 4 bytes starting at address 8 we get: {\tt 98 ba dc fe}.
This is a byte-reversed version of the data word {\tt 0xFEDCBA98}.
This byte reversal represents the proper way to supply the bytes as a
string, since a little endian machine lists the least significant byte
first.  Note also that it only generated two of the four bytes at the end with
value {\tt 00}.  Had we not added this padding, {\sc objdump} gets
even more confused and does not emit all of the bytes we want.

Finally, we can read off the byte sequence for our code (omitting the
final 0's) as:
\begin{tty}
\input{example.txt}
\end{tty}
\end{document}
